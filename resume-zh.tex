
% Chinese version
\documentclass[zh]{resume}

% File information shown at the footer of the last page
\fileinfo{%
  \faCopyright{} 2018--2019 Chaoyang LUO,
  \creativecommons{by}{4.0},
  \githublink{xiaoluolorn}{resume},
  \faCalendarAlt{} \today
}

\name{朝阳}{罗}

\taglineicon{\faBinoculars}
\socialinfo{
  \mobile{185-1968-9255}
  \email{luochaoyang@live.cn}
  \github{xiaoluolorn} \\
  \university{中国社会科学院大学}
  \degree{经济学 \textbullet 博士} \\
  \address{北京}
  \home{河南 \textbullet 开封}
  \birthday{1990-03-04}
}
\photo{7em}{MG.jpg}
\begin{document}
\makeheader

%======================================================================
% Summary & Objectives
%======================================================================
{\onehalfspacing\hspace{2em}
数量经济学专业(宏观经济政策评价方向)博士研究生,有扎实的经济学、数学与统计学基础,
擅长经济数据建模与分析,熟练掌握  Matlab、 Python、 R 和 Stata 语言编程,并
在 \link{https://github.com/xiaoluolorn}{GitHub} 上分享多个项目。
\par} 

%======================================================================
% Competences / Skills & Languages
\sectionTitle{技能和语言}{\faWrench}
%======================================================================
\begin{competences}
  \comptence{编程语言}{%
     Matlab, Python, R 和 Stata
  }
  \comptence{研究工具}{%
    MySQL, Microsoft office Suites, \LaTeX
  }
  \comptence{数据分析}{%
    R, Pandas, Stata; Matplotlib, ggplot2, Origin Pro
  }
  \comptence{\icon{\faLanguage} 语言}{
    \textbf{英语} --- 读写(优良), 听说(日常交流)
  }
\end{competences}

%======================================================================
\sectionTitle{教育背景}{\faGraduationCap}
%======================================================================
\begin{educations}
  \education%
    {2017.09}%
    {中国社会科学院大学}%
    {数量经济与技术经济研究所}%
    {数量经济学(金融风险方向)}%
    {经济学博士(博士研究生,在读,预计 2020 年上半年毕业)}

  \separator{0.5em}
  \education%
    {2014.09}%
    [2017.06]%
    {河南大学}%
    {经济学院}%
    {数量经济学(环境经济方向)}%
    {经济学硕士}

  \separator{0.5em}
  \education%
    {2009.09}%
    [2013.06]%
    {河南农业大学}%
    {工商管理系}%
    {工商管理}%
    {管理学学士}
\end{educations}

%======================================================================
\sectionTitle{计算机技能}{\faCode}
%======================================================================
\begin{itemize}
  \item 自写Matlab程序,将Geoda空间权重文件转变成Matlab空间权重矩阵。
  \item 自写R程序,计算中国公司债券到期收益率问题。
\end{itemize}
%======================================================================
\sectionTitle{实习经历}{\faBriefcase}
%======================================================================
\begin{experiences}
  \experience
    [2018.12]%
    {2019.10 }%
    {新闻数据搜集 \@. 国务院研究中心}%
    [\begin{itemize}
      \item 从多种渠道搜集我国宏观经济运行情况相关新闻。
    \end{itemize}]%

\end{experiences}
%======================================================================
\sectionTitle{科研成果}{\faAtom}
%======================================================================
\begin{itemize}
  \small
  \item 参与项目:
    \enquote{我国经济增长潜力及周期规律研究} (项目支持: 国家发展改革委员会, 主要参与人员).
  \item \textbf{罗朝阳}, 李雪松,
    \enquote{\it 产业结构升级、技术进步与中国能源效率,}
    2019.01, 经济问题探索 (CSSCI)
  \item 李雪松,\textbf{罗朝阳},
    \enquote{\it 金融周期、美联储加息与金融危机,}
    2019.10, 财贸经济 (CSSCI)
  \item \textbf{罗朝阳}, 李雪松,
    \enquote{\it 金融周期、生产率与企业债券违约,}
    2019, 经济管理 (外审中; CSSCI)
\end{itemize}

%======================================================================
\sectionTitle{获奖及证书}{\faAward}
%======================================================================
\begin{entries}
  \entry{2017年}%
    {中国社会科学院大学优秀研究生}
  \entry{2018年}%
    {中国社会科学院大学优秀研究生}
\end{entries}

\end{document}
